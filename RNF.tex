\documentclass[10pt,a4paper]{article}

%
% Recommended way of compilation: XeLaTeX
%

\usepackage{ifxetex}
\ifxetex
	\usepackage{fontenc}
	\usepackage{polyglossia}
	\setdefaultlanguage[variant=usmax]{english}
\else
	\usepackage[utf8]{inputenc}
\fi


\usepackage{verbatimbox}

\usepackage{framed}
\usepackage{enumerate}
\usepackage[pdfborder={0 0 0},hyperfootnotes=false]{hyperref}
\usepackage{tikz}
\usepackage{subcaption}
\usepackage{enumerate}
\usepackage{hyperref}

\usetikzlibrary{arrows,shapes,snakes,backgrounds,petri}

\title{Read Naming Format Specification (draft)}
\author{Karel Břinda \and Valentina Boeva \and Gregory Kucherov}

\addtolength{\textwidth}{5.4cm}
\addtolength{\hoffset}{-2.7cm}
\addtolength{\textheight}{5cm}
\addtolength{\voffset}{-3cm}

\newcommand{\re}[1]{\framebox{\mbox{\texttt{#1}}}}
\newcommand{\mre}[1]{\hspace{0.5cm}\textbf{Matching regular expression:} \re{#1}\smallskip}

\newcommand\todocit[1]{{\todo[inline,color=green!40]{#1}}}
\newcommand\todofix[1]{{\todo[inline,color=blue!40]{#1}}}

\newcommand{\RNF}{\textsc{Rnf}}
\newcommand{\SAM}{\textsc{Sam}}
\newcommand{\FASTA}{{\textsc{Fasta}}}
\newcommand{\FASTQ}{{\textsc{Fastq}}}
\newcommand{\NGS}{\textsc{Ngs}}
\newcommand{\ROC}{\textsc{Roc}}
\newcommand{\TXT}{\textsc{Txt}}
\newcommand{\HTML}{\textsc{Html}}
\newcommand{\CIGAR}{\textsc{Cigar}}

\newcommand{\SAMTOOLS}{{\textsc{SamTools}}}

\begin{document}

\maketitle


\begin{abstract}
	This document provides a standard for
	naming simulated Next-Generation Sequencing (\NGS) reads
	in order to make read simulators and mapper evaluation tools generally inter-compatible.
\end{abstract}


\bigskip

\begin{figure}[h]
\centering
\begin{tikzpicture}[node distance=1.3cm,>=stealth',bend angle=45,auto]

	\tikzstyle{label}=
		[rectangle,thick,fill=orange!50,minimum size=4mm,text width=3cm,align=center]

	\tikzstyle{file}=
		[rectangle,thick,draw=black!75,fill=blue!20,minimum size=12mm,text width=1.8cm,
		align=center]

	\tikzstyle{program}=
		[rectangle,thick,draw=black!75,fill=black!20,minimum size=12mm,align=center,text width=1.8cm]


	\begin{scope}
		\node at (4,1.5) [label] (label1)
			{\textbf{READ\\SIMULATION}};

		\node at (0,0) [file] (s1)
			{Reference genome $1$\\ {\scriptsize [\FASTA/\ldots]}};

		\node at (0,-2.0) [file] (s2)
			{Reference genome $2$\\ {\scriptsize [\FASTA/\ldots]}};

		\node at (0,-3.4) (ss)
			{\Large\textbf{\vdots}};

		\node at (0,-5.0) [file] (sn)
			{Reference genome $n$\\ {\scriptsize [\FASTA/\ldots]}};

		\node at (3,0) [program] (sim)
			{Read \\ simulator}
				edge [pre] (s1)
				edge [pre] (s2)
				edge [pre] (sn);

		\node at (7,0) [file] (fq)
			{Reads \\ {\scriptsize [{\FASTQ}]}}
				edge [pre] node[auto,swap,text width=2cm,align=center,color=red] {\footnotesize \RNF\\ encoding} (sim);

	\end{scope}

	\begin{scope}[xshift=13cm]
		\node at (0,1.5) [label] (label2)
			{\textbf{ALIGNMENT\\EVALUATION}};

		\node at (0,0) [file] (sam)
			{Alignment \\ {\scriptsize{[\SAM]}}};

		\node at (0,-3) [program] (eval)
			{Mapper evaluation tool}
				edge [pre] node[auto,swap,text width=2cm,align=left,color=red]
					{\footnotesize \RNF\\decoding} (sam);

		\node at (0,-5) [file] (report)
			{Report\\ {\scriptsize [\ROC/\TXT/ \HTML/\ldots]}}
				edge [pre] (eval);
	\end{scope}

	\draw[-to,thick,snake=snake,segment amplitude=.4mm,
		 segment length=2mm,line after snake=1mm]
		([xshift=15mm]fq -| fq) -- ([xshift=-15mm]sam -| sam)
		node [above=1mm,midway,text width=3cm,text centered]
		{read mapping};

	\begin{pgfonlayer}{background}
	\filldraw [line width=4mm,join=round,black!5]
		(s1.north -| fq.east) rectangle (sn.south -| s1.west)
		(sam.north	-| report.east)	rectangle (report.south	-| report.west);
	\end{pgfonlayer}
\end{tikzpicture}
\caption{
	Evaluation of mappers of \NGS\ reads using an \RNF-compatible read simulator
	and an \RNF-compatible alignment evaluation tool.
	Reads are simulated from one or more genomes (some of them possibly random), mixed together.
	After aligning them back to the reference sequence, alignments are
	assessed by an alignment evaluation tool, which subsequently creates overall reports.
}
\end{figure}


%%
%%
%%

\section{Terminologies and concepts}

\begin{description}
	\item \textbf{Read tuple} 
	A tuple of sequences (possibly overlapping) obtained from a sequencing machine from a single fragment of DNA.

	\item \textbf{Reads} 
	Members of a {\em read tuple}. For example, every ``paired-end read'' is a {\em read tuple} and both of its ``ends'' are individual {\em reads} in our notation.

	\item \textbf{Segments}
	Substrings of a {\em read} which are spatially distinct in the reference. They correspond to individual lines in a \SAM\ file. Thus, each {\em read} has an
	associated chain of {\em segments} and we associate
	a {\em read tuple} with {\em segments} of all its {\em reads}.
		
	Examples:
	\begin{enumerate}[-]
		\item A ``single-end read'' consists of a single {\em read} with a single {\em segment}
		unless it comes from a region with genomic rearrangment
		\item A ``paired-end read'' or a ``mate-pair read'' consists
		of two {\em reads}, each with one {\em segment} (under the same condition).
		\item A ``strobe read'' consists of several {\em reads}.
		\item A chimeric {\em read} (i.e., read corresponding to a genomic fusion, a long deletion, or a translocation) has at least two {\em segments}.
	\end{enumerate}

%	\item \textbf{Read tuple name} 
	

	\item \textbf{Simulator of NGS reads}
	A program which creates artificial simulated reads from one
	or more (possibly random) reference genomes.
	
		
	\item \textbf{Evaluation tool of NGS mappers} 
	A program which evaluates alignments of simulated \NGS\ reads with known original
	genomic positions. It assesses if each individual read is aligned correctly. 	Finally it usually creates overall statistics.

	\item \textbf{$1$-based coordinate system}
	A coordinate system where the first position has number $1$
	and intervals are closed (the same system is used by the \SAM\ format).
\end{description}


\begin{figure}[!tpb]
\centering

%            0         1          2         3        
\begin{subfigure}{1.0\linewidth}
\centering
\begin{verbbox}[\ ]
Coor        12345678901234-5678901234567890123456789
                                                    
Source 1 - reference genome                        
chr 1       ATGTTAGATAA-GATAGCTGTGCTAGTAGGCAGTCAGCCC
chr 2       ttcttctggaa-gaccttctcctcctgcaaataaa     
                                                    
Source 2 - generator of random sequences                  
                                                    
READS:                                              
r001        ATG-TAGATA ->                           
r002/1         TTAGATAACGA ->                       
r002/2                                  <- TCAG-CGGG
r003/1                               tgcaaataa ->
r003/2              gaa-gacc-t ->                    
r004                       ATAGCT............TCAG ->
r005                                 GTAGG ->
             <- agacctt                           
                        <- TCGACACG   
r006       ATATCACATCATTAGACACTA
\end{verbbox}
\fbox{
\theverbbox
}

\caption{Simulated reads}
\end{subfigure}


\begin{subfigure}{1.0\linewidth}
\centering
\begin{tabular}{c|p{12.0cm}|c}
 \textbf{r. tuple} & \textbf{LRN} & \textbf{SRN} \\\hline
 r001
 	& \texttt{sim\_\_1\_\_(1,1,F,01,10)\_\_[single\_end]}
 	& \texttt{\#1}
 \\\hline
 r002
 	& \texttt{sim\_\_2\_\_(1,1,F,04,14),(1,1,R,31,39)\_\_[paired\_end]}
 	& \texttt{\#2}
 \\\hline
 r003
 	& \texttt{sim\_\_3\_\_(1,2,F,09,17),(1,2,F,25,33)\_\_[mate\_pair]}
 	& \texttt{\#3}
 \\\hline
 r004
 	& \texttt{sim\_\_4\_\_(1,1,F,15,36)\_\_[spliced],{}C:[6=12N4=]}
   	& \texttt{\#4}
 \\\hline
 r005
 	& \texttt{sim\_\_5\_\_(1,1,R,15,22),(1,1,F,25,29),(1,2,R,05,11)\_\_[chimeric]}
 	& \texttt{\#5}
 \\\hline
 r006
 	& \texttt{rnd\_\_6\_\_(2,0,N,00,00)\_\_[random]}
 	& \texttt{\#6}\\
\end{tabular}
\caption{Long and short read names}
\end{subfigure}
\caption{Example of simulated {\em read tuples} and their corresponding
Long Read Names and Short Read Names which are used
in the final \textsc{Fastq} files.
}
\end{figure}


%%
%%
%%

\section{Read tuple names}

To every {\em read tuple}, we assign two names: {Short read name} ({SRN}) and {Long read name} ({LRN}).

\medskip

{SRN} contains a hexademical unique {\em read tuple} ID prefixed by `\texttt{\#}'.

\medskip

{LRN} consists of
four parts delimited by double-underscore:
\begin{enumerate}[i)]
	\item a prefix (possibly containing expressive 
information for a user or a particular string
for sorting or randomization of order of tuples),
	\item the {\em read tuple} ID,
	\item information about origins of all {\em segments} that constitute {\em reads} of the {\em tuple},
	\item a suffix containing arbitrary comments or extensions (for holding additional information).
\end{enumerate}


\medskip

Preferred final read names are {LRNs}. If an {LRN} exceeds $255$ (maximum allowed read length in 
\SAM), 
{SRNs} are used instead and a SRN--LRN correspondence file must be created. 



%%
%%

\subsection{Read tuple ID}

It is a positive integer, which is unique within a single file with genomic data.
These IDs are assigned continuously from $1$. Zero is reserved for ``not available'' (such value should be used only temporarily).


%%
%%

\subsection{SRN -- Short Read Name}
\mre{\char92\char35([0-9a-f]+)}.

{SRN} consists of {\em read tuple} ID prefixed by
`\texttt{\#}'.
It is displayed as zero padded hexadecimals in lowercase such 
that all {SRN}s share the same string
length within a single file.


%%
%%

\subsection{LRN -- Long Read Name}
\mre{\char94([\char33-\char63\char65-\char94\char96-\char126]*)\_\_([0-9a-f]+)\_\_([\char33-\char63\char65-\char94\char96-\char126]+)\_\_([\char33-\char63\char65-\char94\char96-\char126]*)\char36}.

{LRN} consists of four double-underscore-delimited parts:
i)~{prefix part},
ii)~{\em read tuple} ID,
iii)~ {segmental part},
iv)~{suffix part}.



%%
%%

\label{prefix_part}
\subsection*{Prefix part}
\mre{\char94[\char33-\char63\char65-\char94\char96-\char126]*\char36}

It can be an empty string, a string containing
expressive ``visual'' information for the user
(e.g., for easy distinguishing random reads from the others),
or a string used for randomization of {\em read tuples} (randomly taken prefix and {\em read tuples} sorted in lexicographical order).

Length of all prefix parts within a single file must be equal.



%%
%%

\label{readid_part}
\subsection*{Read tuple ID part}
\mre{\char94[0-9a-f]+\char36}

It displays {\em read tuple} ID as hexadecimals
in lowercase. All {\em read tuple} ID parts are zero padded such that
they all share the same string length within a single file.


%%
%%

\subsection*{Segmental part}\label{segmental_part}

\mre{\char94(?:(\char92([0-9FRN,]*\char92))(?:,(?!\char36)|\char36))+\char36}

{Segmental part} consists of one or more comma-delimited {segments}.

%

\subsubsection*{Segment}

\mre{\char94\char92(([0-9]+),([0-9]+),([FRN]),([0-9]+),([0-9]+)\char92)\char36}

Every {segment} is parenthesized and consists
of five comma-delimited values:
i) {genome ID},
ii) {chromosome ID},
iii) {direction},
iv) {leftmost coordinate}, and
v) {rightmost coordinate}.

\medskip

\begin{tabular}{|p{3.4cm}|p{11.0cm}|}
\hline
	{Genome ID} &
		ID (positive integer) of the source genome
		(a randomly generated genome, a genome saved in a \FASTA\ file,
		etc.) or zero for ``not available''. 
		\newline\newline
		All numbers of genomes are displayed as decimals and they
		are zero padded such that
		they all share the same string length within a single file. 
		\\\hline
	{Chromosome ID} &
		ID (positive integer) of the source chromosome
		or zero for ``not available''.
		\newline
		\newline
		All numbers of chromosomes are displayed as decimals and they are zero padded such that they all share the same string length within a single file.
		\newline
		\newline
		IDs are assigned continuously from $1$,
		order chromosomes is the same as in
		the file, where the genome is saved.
		In case of a random genome, zero should be used.
		\\\hline
	{Direction} &
		Direction in the reference genome.
		\newline
		\newline
		`F' = forward direction \newline
		`R' = reverse direction \newline
		`N' = not available
		\newline
		\newline
		For random reads, `N' should be used.
		\\\hline
	{Leftmost coordinate} &
		The leftmost
		coordinate of the segment in the reference in 1-based coordinate system or zero for ``not available''. 
		\\\hline
	{Rightmost coordinate} & 
		The rightmost
		coordinate of the segment in the reference in 1-based coordinate system or zero for ``not available''. 
		\\\hline
\end{tabular}



%%
%%

\label{suffix_part}
\subsection*{Suffix part}
\mre{\char94(?:((?:[a-zA-Z0-9]+:)\char123{}0,1\char125)\char92[([\char33-\char63\char65-\char90\char92\char92\char94\char96-\char126]*)\char92](?:,(?!\char36)|\char36))+\char36}

It contains arbitrary number of comma-delimited
{comments} and {extensions} in
any order.


%%

\subsubsection*{Comment}
\mre{\char94[([\char33-\char63\char65-\char90\char92\char92\char94\char96-\char126]*)]\char36}

{Comments} are displayed as square-bracketed strings.
They can contain, e.g., information
about the simulated technology or the program used for
simulation.

%%

\subsubsection*{Extension}
\mre{\char94([A-Za-z0-9]+):\char92[([\char33-\char63\char65-\char90\char92\char92\char94\char96-\char126]*)\char92]\char36}

An {extension} consist of an {extension's code}, a colon, and
a square-bracketed {extension's content}.
Extensions can supplement the basic set of information provided in segmental part.
Some of them are part of this standard,
see Section~\ref{sec:extensions}.


%%
%%
%%

\section{SRN--LRN correspondence file}
\label{sec:correspondence_file}

To encode information about correspondence between
SRN and LRN,
a special file is created. Its file name is formed of
prefix of the \FASTQ\ file(s) and
\texttt{.sl} suffix.

\medskip

Examples:
\smallskip

\begin{tabular}{l|l}
	Read files & SRN-LRN correspondence file \\\hline
	\texttt{reads\_se.fq} & \texttt{reads\_se.sl} \\
	\texttt{reads\_se.fastq} & \texttt{reads\_se.sl} \\
	\texttt{reads\_pe.1.fq}, \texttt{reads\_pe.2.fq} & \texttt{reads\_pe.sl}
\end{tabular}

\smallskip

It is a
tab delimited file with two columns (containing SRN and the corresponding LRN). File is sorted by {\em read tuple} ID.


%%
%%
%%

\section{Extensions}
\label{sec:extensions}
\appendix

Extensions can supplement the basic set of information provided in the segmental part (Section \ref{segmental_part}).

\subsection*{C -- CIGAR strings}

\subsubsection*{Extension's code}

\hspace{0.5cm} C

\subsubsection*{Extension's content}

\mre{\char94(?:([0-9]+[=XIDNSHPM]+)(?:,(?!\char36)|\char36))+\char36}


\subsubsection*{Specification}

The extension can be used to encode edit operations 
using \CIGAR\  (Compact Idiosyncratic Gapped Alignment Report) strings.

\medskip

\begin{table}[h]
\centering
\caption*{Supported operations:}
\begin{tabular}{|l|l|p{8cm}|}
	\hline
	letter & operation & comment 
	\\\hline
	\texttt{=} & match & \\
	\texttt{X} & mismatch & \\
	\texttt{I} & insertion & \\
	\texttt{D} & deletion &	\\
	\texttt{N} & skipping bases & skipping intron regions in spliced mapping \\
	\texttt{S} & soft clipping &
		for cutting unaligned prefixes and suffixes \\
	\texttt{H} & hard clipping &
		for cutting unaligned prefixes and suffixes \\
	\texttt{P} & padding & unused padding in padded reference\\
	\texttt{M} & match or mismatch	& DEPRECATED, reserved for situations when distinguishing \texttt{X} vs. \texttt{=}
is impossible
	\\\hline
\end{tabular}
\end{table}

\medskip

\CIGAR\ strings should be provided in the same order as their corresponding segments in the {segmental part} (Section \ref{segmental_part}). Adjacent edit operations should be different.

\subsubsection*{Example}

\begin{framed}\small
\begin{verbatim}
	demonstration__004__(1,1,F,16,40)__C:[6=14N5=],[spliced_read]
\end{verbatim}
\end{framed}




%%
%%
%%

\begin{thebibliography}{}
	\bibitem{samtools}
	Li, H. \textit{et al.} (2009)
	The Sequence Alignment/Map format and SAMtools.
	\textit{Bioinformatics} \textbf{25}(16): 2078--2079.
\end{thebibliography}


\end{document}

